\documentclass[12pt,a4]{article}


\title{Detecção e extração de textos em imgens de redação, para avaliação de competências do ENEM}
\date{}

\begin{document}
\begin{titlepage}
\begin{center}
\maketitle
\textbf{Alunos}: Lucas Francisco Correia, Wilson Neto,  Ismael Cesar \\
\textbf{Disciplina}: Processamento de imagens
\end{center}
\end{titlepage}

\section{Introdução}
\label{sec:introdcao}
	Extrair textos de imagens pode ser uma atividade útil para realisação de tarefas como, tradução ou até mesmo digitalização de documentos.
	Porém este trabalho tem um propósito específico de extração de textos de folhas de redação da prova do Exame nacional do Ensino Médio (ENEM).
	Uma prova de nível nacional como o ENEM é realizada para centenas de milhares de candidatos, logo centenas de milhares de redações precisam ser avaliadas.
	Extração de texto de folhas de redação para avaliação automática tem pode reduzir em muito o tempo necessário para que todas as redações sejam avaliadas bem como tornar desnecessário o uso de avaliadores, diminuindo os custos para o estado. 
	Este trabalho foca em metodologias utilizadas para extração de textos em folhas de redações do ENEM.
	Como base foi utilizado o conjunto de $31$ redações disponibilizados por estudantes na plataforma UOL. 
	

\section{Medotologia}
\label{sec:metodologia}
	A primeira etapa para extração de texto trata-se do préprocessamento. 
	Como as imagens de redação já vem em um formato padronizado fora utilizada somente a técnica de limiarização adaptativa gaussiana. 
	Onde o tipo de binarização escolhida foi a binarização invertida. 
	O tamanho do bloco utilizado foi $21$.
	E valor da constante utilizada no método de limiarização gaussiana foi $15$.

\bibliography{bibliografia}
\end{document}